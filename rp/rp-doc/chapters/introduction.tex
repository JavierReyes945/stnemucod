%%%
%  File: introduction.tex
%  Project: rp-doc
%  Author: Javier Reyes (javier.reyes.g@gmail.com)
%  Created on: 08.09.2018
%  
%  Modified By: Javier Reyes
%  Last Modified: 10.09.2018
%  
%  MIT License
%  
%  Copyright (c) 2018 Javier Reyes
%  
%  Permission is hereby granted, free of charge, to any person obtaining a copy of
%  this software and associated documentation files (the "Software"), to deal in
%  the Software without restriction, including without limitation the rights to
%  use, copy, modify, merge, publish, distribute, sublicense, and/or sell copies
%  of the Software, and to permit persons to whom the Software is furnished to do
%  so, subject to the following conditions:
%  
%  The above copyright notice and this permission notice shall be included in all
%  copies or substantial portions of the Software.
%  
%  THE SOFTWARE IS PROVIDED "AS IS", WITHOUT WARRANTY OF ANY KIND, EXPRESS OR
%  IMPLIED, INCLUDING BUT NOT LIMITED TO THE WARRANTIES OF MERCHANTABILITY,
%  FITNESS FOR A PARTICULAR PURPOSE AND NONINFRINGEMENT. IN NO EVENT SHALL THE
%  AUTHORS OR COPYRIGHT HOLDERS BE LIABLE FOR ANY CLAIM, DAMAGES OR OTHER
%  LIABILITY, WHETHER IN AN ACTION OF CONTRACT, TORT OR OTHERWISE, ARISING FROM,
%  OUT OF OR IN CONNECTION WITH THE SOFTWARE OR THE USE OR OTHER DEALINGS IN THE
%  SOFTWARE.
%%%

\chapter*{Introduction}

This work is part of a large project from IDiAL (\textit{Institut für die Digitalisierung von
Arbeits- und Lebens­welten}), and its progress is based on the linking and cooperation of bachelor
and master student projects. DAEbot project is explained in more detail on chapter \ref{chapter1}.

Following the project guideline, the development of a single component of the structural model can
be handled almost independently. The results can be focused on the convenience and efectiveness of
the platform and the integration with the rest of the system.

The purpose of this document is present the workflow for the design and implementation of a usable
robotic application on a Zynqberry platform as a reflective operator+ module, and evaluate it in
terms of technological capabilities and constraints for the robot system.

The document consists of this introductorial chapter, and five content chapters. The first chapter
presents an overview of the robotic platform system called DAEbot with its architectural model and
components, and then focuses on the specific component along with a description of the hardware
platform that is used. The second chapter presents the workflow for the hardware design and
implementation for the selected platform, introducing the necessary tools. The third chapter
presents the necessary OS configuration and build, in order to have a functional embedded system
platform. The fourth chapter presents the corresponding software development workflow, and all
adjacent considerations for its execution. Finally on the fifth chapter, the results and
experiences of the work are presented as a manner of legacy for future development, and conclusions
to the whole operator plus DAEbot project.
