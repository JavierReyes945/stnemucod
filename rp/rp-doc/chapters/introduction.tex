% Introduction (from main tex file)
% Research Project
% Author: Javier Reyes

\chapter*{Introduction}

This work is part of a large project from IDiAL (\textit{Institut für die Digitalisierung von
Arbeits- und Lebens­welten}), and its progress is based on the linking and cooperation of bachelor
and master student projects.

The DAEbot project consists on ... % TODO: Find official daebot notes

Following the project guideline, the development of a single component of the structural model can
be handled almost independently. The results can be focused on the convenience and efectiveness of
the platform and the integration with the rest of the system.

The purpose of this document is present the workflow for the design and implementation of a usable
robotic application on a Zynqberry platform as a reflective operator+ module, and evaluate it in
terms of technological capabilities and constraints for the robot system.

The document consists of this introductorial chapter, and four content chapters. The second chapter
presents an overview of the robotic platform system called DAEbot with its architectural model and
components, and then focuses on the specific component along with a description of the hardware
platform that is used. The third chapter presents the workflow for the hardware design and
implementation for the selected platform, introducing the necessary tools. The fourth chapter
presents the corresponding software development workflow, including the necessary OS creation and
all adjacent considerations for its execution. Finally on the fifth chapter, the results and
experiences of the work are presented as a manner of legacy for future development, and conclusions
to the whole DAEbot project.
