%%%
%  File: appendix1.tex
%  Project: rp-doc
%  Author: Javier Reyes (javier.reyes.g@gmail.com)
%  Created on: 08.09.2018
%  
%  Modified By: Javier Reyes
%  Last Modified: 10.09.2018
%  
%  MIT License
%  
%  Copyright (c) 2018 Javier Reyes
%  
%  Permission is hereby granted, free of charge, to any person obtaining a copy of
%  this software and associated documentation files (the "Software"), to deal in
%  the Software without restriction, including without limitation the rights to
%  use, copy, modify, merge, publish, distribute, sublicense, and/or sell copies
%  of the Software, and to permit persons to whom the Software is furnished to do
%  so, subject to the following conditions:
%  
%  The above copyright notice and this permission notice shall be included in all
%  copies or substantial portions of the Software.
%  
%  THE SOFTWARE IS PROVIDED "AS IS", WITHOUT WARRANTY OF ANY KIND, EXPRESS OR
%  IMPLIED, INCLUDING BUT NOT LIMITED TO THE WARRANTIES OF MERCHANTABILITY,
%  FITNESS FOR A PARTICULAR PURPOSE AND NONINFRINGEMENT. IN NO EVENT SHALL THE
%  AUTHORS OR COPYRIGHT HOLDERS BE LIABLE FOR ANY CLAIM, DAMAGES OR OTHER
%  LIABILITY, WHETHER IN AN ACTION OF CONTRACT, TORT OR OTHERWISE, ARISING FROM,
%  OUT OF OR IN CONNECTION WITH THE SOFTWARE OR THE USE OR OTHER DEALINGS IN THE
%  SOFTWARE.
%%%

\chapter{Linux in the Zynqberry} \label{appen1}

The following procedure shows the actual steps followed to build and boot an embedded Linux image
in the Zynqberry 726 board, based on the different material available from the manufacturer of the
board and the Zynq device.

Every board provided by Trenz Electronics can have different procedures, due to technical
characteristics or limitations. This document will be only valid for the model TE0727-02R on which
the work was tested. The tools and software for the hardware design (Xilinx Vivado), software
development (Xilinx SDK) and OS image building (Xilinx Petalinux) are valid only for the version
2017.4. It has been observed that every new release of this tools imply a quite substancial change
in the commands, or the workflow. There is no guaranteed backwards or forward compatibility.

In order to define the environment, the following requisites are established:

\begin{itemize}
	\item The board TE 0726-02R from Trenz Electronics, together with a USB cable.
	\item An SD card (appropiate for boot purposes, speed class 10).
	\item A working station complying with the minimum installation requirements for the Vivado
	Design Suite 2017.4 (See \cite{UG973}).
	\item A working station complying with the minimum installation requirements for the Petalinux
	SDK 2017.4 (See \cite{UG1144}).
	\item Internet connectivity during the Linux image build process.
\end{itemize}

% TODO: Complete chapter

\section{Hardware Project}

\section{Embedded Linux image build}

\section{Boot up process}
