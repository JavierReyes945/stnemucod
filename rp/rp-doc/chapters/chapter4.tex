%%%
%  File: chapter4.tex
%  Project: rp-doc
%  Author: Javier Reyes (javier.reyes.g@gmail.com)
%  Created on: 09.09.2018
%  
%  Modified By: Javier Reyes
%  Last Modified: 10.09.2018
%  
%  MIT License
%  
%  Copyright (c) 2018 Javier Reyes
%  
%  Permission is hereby granted, free of charge, to any person obtaining a copy of
%  this software and associated documentation files (the "Software"), to deal in
%  the Software without restriction, including without limitation the rights to
%  use, copy, modify, merge, publish, distribute, sublicense, and/or sell copies
%  of the Software, and to permit persons to whom the Software is furnished to do
%  so, subject to the following conditions:
%  
%  The above copyright notice and this permission notice shall be included in all
%  copies or substantial portions of the Software.
%  
%  THE SOFTWARE IS PROVIDED "AS IS", WITHOUT WARRANTY OF ANY KIND, EXPRESS OR
%  IMPLIED, INCLUDING BUT NOT LIMITED TO THE WARRANTIES OF MERCHANTABILITY,
%  FITNESS FOR A PARTICULAR PURPOSE AND NONINFRINGEMENT. IN NO EVENT SHALL THE
%  AUTHORS OR COPYRIGHT HOLDERS BE LIABLE FOR ANY CLAIM, DAMAGES OR OTHER
%  LIABILITY, WHETHER IN AN ACTION OF CONTRACT, TORT OR OTHERWISE, ARISING FROM,
%  OUT OF OR IN CONNECTION WITH THE SOFTWARE OR THE USE OR OTHER DEALINGS IN THE
%  SOFTWARE.
%%%

\chapter{Software Application} \label{software-application}

The current application is in WIP (Work In Progress) state, as there are technical challenges to
overcome. Esentially, the application is coded in C (with an option created in C++ for testing), and
built in the Xilinx SDK via the cross-compilers that Xilinx provide for the ARM7 architecture.

\section{Architecture}

The program follows a simple principle, with a single thread execution, but can be expanded with a
thread definition for the handling of the hardware control.

\begin{figure}[htp]
	\centering
	\includegraphics[width=0.7\textwidth]{activity-diag-eth.png}
	\caption{Activity diagram for server application in Zynqberry.}
	\label{fig:activity-diag-eth}
\end{figure}

Given the expected estandard workflow for developing Linux applications for a Zynq device, the
application should be included in the image building process (with the
\texttt{petalinux-create -t apps --name myApp}), so that it can be installed properly in the system.
The previous workflow expects an INITRAMFS, which is not the case we used. Therefore, we had to
manually copy the application executable into the rootfs of the Zynqberry. The simplest way is to
use \texttt{scp} command with the Zynqberry connected to the local network.

As mentioned in chapter \ref{chapter1}, the Zynqberry device should be able to receive a video
stream from the Cam Controller device. Is it then necessary that the application acted as a server,
for the client application that was developed for the Cam Controller device, and described in
\cite{Poliakov2018}.

In order to handle a hardware processing functionality in the PL part of the Zynq device, a IP core
module needs to be included (as described in \ref{custom-ip-core}). This makes also necessary to add
a software driver for the IP core handling, which provides the methods for read/write operations to
the memory mapped registers of the IP Core. This part is not shown in the diagram, as it is neither
functional nor validated. The documentation provided by Xilinx describes the usage of base functions
in the BSP, but for standalone platform, which makes this part of the work more complex.

\section{CAN Communication}

As the communication with the rest of the devices is performed via CAN, the image was built with the
CAN module kernel enabled. The usage of the CAN device is still in debug state, as the recommended
flow is to use the CANutils device drivers, available from the apt repositories for the Xilinx
Linux, but this has not worked as expected. It should be noted that the Standalone usage of the CAN
module in the Zynqberry works flawlessly, and that the material presented by Xilinx related to
the usage of CAN in the Linux image is contradictory to the configuration present in the petalinux
template provided by the manufacturer.
